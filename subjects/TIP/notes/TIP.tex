%%This is a very basic article template.
%%There is just one section and two subsections.
\documentclass[a4paper,11pt]{article}
\usepackage[pdfborder= 0 0 0 1]{hyperref}
\usepackage{graphicx}
\usepackage{appendix}
\usepackage{tabularx}
%%\usepackage{draftwatermark}
%%\SetWatermarkLightness{0.9}
%%\SetWatermarkScale{5}

\begin{document}
\title{\textbf{{\LARGE TIP notes \\2012/13}}}
\author{Carlos Gil Soriano\\UAM-HPCN\\
\hypersetup{
  colorlinks   =  true,
  urlcolor     =  blue
  pdftitle     = {TIP notes 2012/2013},
  pdfauthor    = {Carlos Gil Soriano},
  pdfsubject   = {TIP notes for the course 2012/2013},
  pdfkeywords  = {Euler, SDF, Brownian, Newton-Rhapson}
}
\href{mailto:gilsoriano@gmail.com}{\textbf{\textit{gilsoriano@gmail.com}}}}
\date{\today}
\maketitle
\thispagestyle{empty}
\begin{figure}[htb]
   \begin{center}
      \includegraphics[scale=1,
      keepaspectratio]{../../../figures/logo/logo-uam.jpg}
   \end{center}
\end{figure}

\begin{abstract}

  This subject introduces the tools for the analysis, modeling, prediction and
  simulation of time series. The models considered include advanced modeling
  techniques, such as non-linear models, non-parametric models, neural networks,
  Bayesian methods, etc. We will also consider learning, control and information
  processing problems in dynamical environments and in the presence of
  uncertainty. The presentation is structured around practical applications that
  involve the analysis of environmental, financial, biological or medical time
  series.

\end{abstract}

\vspace{2cm}
\begin{center}
\begin{tabular}{|p{2.5cm}|p{3.5cm}|p{3.5cm}|}
\hline
\multicolumn{3}{|c|}{\textbf{Revision history}}\\
\hline
\hline
\textbf{HDL version} & \textbf{Module} & \textbf{Date}\\
\hline
0.1 & First notes & \today\\
\hline
\end{tabular}
\end{center}

\pagebreak

\pagenumbering{roman}
\setcounter{page}{1}
\pagebreak

\setcounter{tocdepth}{3}
\tableofcontents
\pagebreak
\listoftables
\listoffigures
\pagebreak

\pagenumbering{arabic}
\setcounter{page}{1}

\pagebreak
\section{Structure}
\begin{center}
\begin{tabularx}{\textwidth}{|X|}
\hline
It is recommended to take a look to the I2C standard before continue reading.\\
\hline
\end{tabularx}
\end{center}

\pagebreak
\subsection{Intended use}

\appendix
\section{Classes}
  \begin{center}
    \begin{tabular}{|p{1.5cm}|p{4cm}|p{5.5cm}|}
      \hline
      \textbf{Session} & \textbf{Date} & \textbf{Contents}\\
      \hline
      2 & January 29, 2013  & Wiener process\\
        & & Ito convention for SDE\\
      \hline
      3 &  January 31, 2013 & Arithmetic Brownian     \\
        &                   & Geometric Brownian      \\
        &                   & Random Number Generator \\
        &                   & Newton-Rhapson          \\
      \hline
    \end{tabular}
  \end{center}

\end{document}
